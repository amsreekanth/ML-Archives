
% Default to the notebook output style

    


% Inherit from the specified cell style.




    
\documentclass[11pt]{article}

    
    
    \usepackage[T1]{fontenc}
    % Nicer default font (+ math font) than Computer Modern for most use cases
    \usepackage{mathpazo}

    % Basic figure setup, for now with no caption control since it's done
    % automatically by Pandoc (which extracts ![](path) syntax from Markdown).
    \usepackage{graphicx}
    % We will generate all images so they have a width \maxwidth. This means
    % that they will get their normal width if they fit onto the page, but
    % are scaled down if they would overflow the margins.
    \makeatletter
    \def\maxwidth{\ifdim\Gin@nat@width>\linewidth\linewidth
    \else\Gin@nat@width\fi}
    \makeatother
    \let\Oldincludegraphics\includegraphics
    % Set max figure width to be 80% of text width, for now hardcoded.
    \renewcommand{\includegraphics}[1]{\Oldincludegraphics[width=.8\maxwidth]{#1}}
    % Ensure that by default, figures have no caption (until we provide a
    % proper Figure object with a Caption API and a way to capture that
    % in the conversion process - todo).
    \usepackage{caption}
    \DeclareCaptionLabelFormat{nolabel}{}
    \captionsetup{labelformat=nolabel}

    \usepackage{adjustbox} % Used to constrain images to a maximum size 
    \usepackage{xcolor} % Allow colors to be defined
    \usepackage{enumerate} % Needed for markdown enumerations to work
    \usepackage{geometry} % Used to adjust the document margins
    \usepackage{amsmath} % Equations
    \usepackage{amssymb} % Equations
    \usepackage{textcomp} % defines textquotesingle
    % Hack from http://tex.stackexchange.com/a/47451/13684:
    \AtBeginDocument{%
        \def\PYZsq{\textquotesingle}% Upright quotes in Pygmentized code
    }
    \usepackage{upquote} % Upright quotes for verbatim code
    \usepackage{eurosym} % defines \euro
    \usepackage[mathletters]{ucs} % Extended unicode (utf-8) support
    \usepackage[utf8x]{inputenc} % Allow utf-8 characters in the tex document
    \usepackage{fancyvrb} % verbatim replacement that allows latex
    \usepackage{grffile} % extends the file name processing of package graphics 
                         % to support a larger range 
    % The hyperref package gives us a pdf with properly built
    % internal navigation ('pdf bookmarks' for the table of contents,
    % internal cross-reference links, web links for URLs, etc.)
    \usepackage{hyperref}
    \usepackage{longtable} % longtable support required by pandoc >1.10
    \usepackage{booktabs}  % table support for pandoc > 1.12.2
    \usepackage[inline]{enumitem} % IRkernel/repr support (it uses the enumerate* environment)
    \usepackage[normalem]{ulem} % ulem is needed to support strikethroughs (\sout)
                                % normalem makes italics be italics, not underlines
    

    
    
    % Colors for the hyperref package
    \definecolor{urlcolor}{rgb}{0,.145,.698}
    \definecolor{linkcolor}{rgb}{.71,0.21,0.01}
    \definecolor{citecolor}{rgb}{.12,.54,.11}

    % ANSI colors
    \definecolor{ansi-black}{HTML}{3E424D}
    \definecolor{ansi-black-intense}{HTML}{282C36}
    \definecolor{ansi-red}{HTML}{E75C58}
    \definecolor{ansi-red-intense}{HTML}{B22B31}
    \definecolor{ansi-green}{HTML}{00A250}
    \definecolor{ansi-green-intense}{HTML}{007427}
    \definecolor{ansi-yellow}{HTML}{DDB62B}
    \definecolor{ansi-yellow-intense}{HTML}{B27D12}
    \definecolor{ansi-blue}{HTML}{208FFB}
    \definecolor{ansi-blue-intense}{HTML}{0065CA}
    \definecolor{ansi-magenta}{HTML}{D160C4}
    \definecolor{ansi-magenta-intense}{HTML}{A03196}
    \definecolor{ansi-cyan}{HTML}{60C6C8}
    \definecolor{ansi-cyan-intense}{HTML}{258F8F}
    \definecolor{ansi-white}{HTML}{C5C1B4}
    \definecolor{ansi-white-intense}{HTML}{A1A6B2}

    % commands and environments needed by pandoc snippets
    % extracted from the output of `pandoc -s`
    \providecommand{\tightlist}{%
      \setlength{\itemsep}{0pt}\setlength{\parskip}{0pt}}
    \DefineVerbatimEnvironment{Highlighting}{Verbatim}{commandchars=\\\{\}}
    % Add ',fontsize=\small' for more characters per line
    \newenvironment{Shaded}{}{}
    \newcommand{\KeywordTok}[1]{\textcolor[rgb]{0.00,0.44,0.13}{\textbf{{#1}}}}
    \newcommand{\DataTypeTok}[1]{\textcolor[rgb]{0.56,0.13,0.00}{{#1}}}
    \newcommand{\DecValTok}[1]{\textcolor[rgb]{0.25,0.63,0.44}{{#1}}}
    \newcommand{\BaseNTok}[1]{\textcolor[rgb]{0.25,0.63,0.44}{{#1}}}
    \newcommand{\FloatTok}[1]{\textcolor[rgb]{0.25,0.63,0.44}{{#1}}}
    \newcommand{\CharTok}[1]{\textcolor[rgb]{0.25,0.44,0.63}{{#1}}}
    \newcommand{\StringTok}[1]{\textcolor[rgb]{0.25,0.44,0.63}{{#1}}}
    \newcommand{\CommentTok}[1]{\textcolor[rgb]{0.38,0.63,0.69}{\textit{{#1}}}}
    \newcommand{\OtherTok}[1]{\textcolor[rgb]{0.00,0.44,0.13}{{#1}}}
    \newcommand{\AlertTok}[1]{\textcolor[rgb]{1.00,0.00,0.00}{\textbf{{#1}}}}
    \newcommand{\FunctionTok}[1]{\textcolor[rgb]{0.02,0.16,0.49}{{#1}}}
    \newcommand{\RegionMarkerTok}[1]{{#1}}
    \newcommand{\ErrorTok}[1]{\textcolor[rgb]{1.00,0.00,0.00}{\textbf{{#1}}}}
    \newcommand{\NormalTok}[1]{{#1}}
    
    % Additional commands for more recent versions of Pandoc
    \newcommand{\ConstantTok}[1]{\textcolor[rgb]{0.53,0.00,0.00}{{#1}}}
    \newcommand{\SpecialCharTok}[1]{\textcolor[rgb]{0.25,0.44,0.63}{{#1}}}
    \newcommand{\VerbatimStringTok}[1]{\textcolor[rgb]{0.25,0.44,0.63}{{#1}}}
    \newcommand{\SpecialStringTok}[1]{\textcolor[rgb]{0.73,0.40,0.53}{{#1}}}
    \newcommand{\ImportTok}[1]{{#1}}
    \newcommand{\DocumentationTok}[1]{\textcolor[rgb]{0.73,0.13,0.13}{\textit{{#1}}}}
    \newcommand{\AnnotationTok}[1]{\textcolor[rgb]{0.38,0.63,0.69}{\textbf{\textit{{#1}}}}}
    \newcommand{\CommentVarTok}[1]{\textcolor[rgb]{0.38,0.63,0.69}{\textbf{\textit{{#1}}}}}
    \newcommand{\VariableTok}[1]{\textcolor[rgb]{0.10,0.09,0.49}{{#1}}}
    \newcommand{\ControlFlowTok}[1]{\textcolor[rgb]{0.00,0.44,0.13}{\textbf{{#1}}}}
    \newcommand{\OperatorTok}[1]{\textcolor[rgb]{0.40,0.40,0.40}{{#1}}}
    \newcommand{\BuiltInTok}[1]{{#1}}
    \newcommand{\ExtensionTok}[1]{{#1}}
    \newcommand{\PreprocessorTok}[1]{\textcolor[rgb]{0.74,0.48,0.00}{{#1}}}
    \newcommand{\AttributeTok}[1]{\textcolor[rgb]{0.49,0.56,0.16}{{#1}}}
    \newcommand{\InformationTok}[1]{\textcolor[rgb]{0.38,0.63,0.69}{\textbf{\textit{{#1}}}}}
    \newcommand{\WarningTok}[1]{\textcolor[rgb]{0.38,0.63,0.69}{\textbf{\textit{{#1}}}}}
    
    
    % Define a nice break command that doesn't care if a line doesn't already
    % exist.
    \def\br{\hspace*{\fill} \\* }
    % Math Jax compatability definitions
    \def\gt{>}
    \def\lt{<}
    % Document parameters
    \title{1HabermanAnalysis}
    
    
    

    % Pygments definitions
    
\makeatletter
\def\PY@reset{\let\PY@it=\relax \let\PY@bf=\relax%
    \let\PY@ul=\relax \let\PY@tc=\relax%
    \let\PY@bc=\relax \let\PY@ff=\relax}
\def\PY@tok#1{\csname PY@tok@#1\endcsname}
\def\PY@toks#1+{\ifx\relax#1\empty\else%
    \PY@tok{#1}\expandafter\PY@toks\fi}
\def\PY@do#1{\PY@bc{\PY@tc{\PY@ul{%
    \PY@it{\PY@bf{\PY@ff{#1}}}}}}}
\def\PY#1#2{\PY@reset\PY@toks#1+\relax+\PY@do{#2}}

\expandafter\def\csname PY@tok@w\endcsname{\def\PY@tc##1{\textcolor[rgb]{0.73,0.73,0.73}{##1}}}
\expandafter\def\csname PY@tok@c\endcsname{\let\PY@it=\textit\def\PY@tc##1{\textcolor[rgb]{0.25,0.50,0.50}{##1}}}
\expandafter\def\csname PY@tok@cp\endcsname{\def\PY@tc##1{\textcolor[rgb]{0.74,0.48,0.00}{##1}}}
\expandafter\def\csname PY@tok@k\endcsname{\let\PY@bf=\textbf\def\PY@tc##1{\textcolor[rgb]{0.00,0.50,0.00}{##1}}}
\expandafter\def\csname PY@tok@kp\endcsname{\def\PY@tc##1{\textcolor[rgb]{0.00,0.50,0.00}{##1}}}
\expandafter\def\csname PY@tok@kt\endcsname{\def\PY@tc##1{\textcolor[rgb]{0.69,0.00,0.25}{##1}}}
\expandafter\def\csname PY@tok@o\endcsname{\def\PY@tc##1{\textcolor[rgb]{0.40,0.40,0.40}{##1}}}
\expandafter\def\csname PY@tok@ow\endcsname{\let\PY@bf=\textbf\def\PY@tc##1{\textcolor[rgb]{0.67,0.13,1.00}{##1}}}
\expandafter\def\csname PY@tok@nb\endcsname{\def\PY@tc##1{\textcolor[rgb]{0.00,0.50,0.00}{##1}}}
\expandafter\def\csname PY@tok@nf\endcsname{\def\PY@tc##1{\textcolor[rgb]{0.00,0.00,1.00}{##1}}}
\expandafter\def\csname PY@tok@nc\endcsname{\let\PY@bf=\textbf\def\PY@tc##1{\textcolor[rgb]{0.00,0.00,1.00}{##1}}}
\expandafter\def\csname PY@tok@nn\endcsname{\let\PY@bf=\textbf\def\PY@tc##1{\textcolor[rgb]{0.00,0.00,1.00}{##1}}}
\expandafter\def\csname PY@tok@ne\endcsname{\let\PY@bf=\textbf\def\PY@tc##1{\textcolor[rgb]{0.82,0.25,0.23}{##1}}}
\expandafter\def\csname PY@tok@nv\endcsname{\def\PY@tc##1{\textcolor[rgb]{0.10,0.09,0.49}{##1}}}
\expandafter\def\csname PY@tok@no\endcsname{\def\PY@tc##1{\textcolor[rgb]{0.53,0.00,0.00}{##1}}}
\expandafter\def\csname PY@tok@nl\endcsname{\def\PY@tc##1{\textcolor[rgb]{0.63,0.63,0.00}{##1}}}
\expandafter\def\csname PY@tok@ni\endcsname{\let\PY@bf=\textbf\def\PY@tc##1{\textcolor[rgb]{0.60,0.60,0.60}{##1}}}
\expandafter\def\csname PY@tok@na\endcsname{\def\PY@tc##1{\textcolor[rgb]{0.49,0.56,0.16}{##1}}}
\expandafter\def\csname PY@tok@nt\endcsname{\let\PY@bf=\textbf\def\PY@tc##1{\textcolor[rgb]{0.00,0.50,0.00}{##1}}}
\expandafter\def\csname PY@tok@nd\endcsname{\def\PY@tc##1{\textcolor[rgb]{0.67,0.13,1.00}{##1}}}
\expandafter\def\csname PY@tok@s\endcsname{\def\PY@tc##1{\textcolor[rgb]{0.73,0.13,0.13}{##1}}}
\expandafter\def\csname PY@tok@sd\endcsname{\let\PY@it=\textit\def\PY@tc##1{\textcolor[rgb]{0.73,0.13,0.13}{##1}}}
\expandafter\def\csname PY@tok@si\endcsname{\let\PY@bf=\textbf\def\PY@tc##1{\textcolor[rgb]{0.73,0.40,0.53}{##1}}}
\expandafter\def\csname PY@tok@se\endcsname{\let\PY@bf=\textbf\def\PY@tc##1{\textcolor[rgb]{0.73,0.40,0.13}{##1}}}
\expandafter\def\csname PY@tok@sr\endcsname{\def\PY@tc##1{\textcolor[rgb]{0.73,0.40,0.53}{##1}}}
\expandafter\def\csname PY@tok@ss\endcsname{\def\PY@tc##1{\textcolor[rgb]{0.10,0.09,0.49}{##1}}}
\expandafter\def\csname PY@tok@sx\endcsname{\def\PY@tc##1{\textcolor[rgb]{0.00,0.50,0.00}{##1}}}
\expandafter\def\csname PY@tok@m\endcsname{\def\PY@tc##1{\textcolor[rgb]{0.40,0.40,0.40}{##1}}}
\expandafter\def\csname PY@tok@gh\endcsname{\let\PY@bf=\textbf\def\PY@tc##1{\textcolor[rgb]{0.00,0.00,0.50}{##1}}}
\expandafter\def\csname PY@tok@gu\endcsname{\let\PY@bf=\textbf\def\PY@tc##1{\textcolor[rgb]{0.50,0.00,0.50}{##1}}}
\expandafter\def\csname PY@tok@gd\endcsname{\def\PY@tc##1{\textcolor[rgb]{0.63,0.00,0.00}{##1}}}
\expandafter\def\csname PY@tok@gi\endcsname{\def\PY@tc##1{\textcolor[rgb]{0.00,0.63,0.00}{##1}}}
\expandafter\def\csname PY@tok@gr\endcsname{\def\PY@tc##1{\textcolor[rgb]{1.00,0.00,0.00}{##1}}}
\expandafter\def\csname PY@tok@ge\endcsname{\let\PY@it=\textit}
\expandafter\def\csname PY@tok@gs\endcsname{\let\PY@bf=\textbf}
\expandafter\def\csname PY@tok@gp\endcsname{\let\PY@bf=\textbf\def\PY@tc##1{\textcolor[rgb]{0.00,0.00,0.50}{##1}}}
\expandafter\def\csname PY@tok@go\endcsname{\def\PY@tc##1{\textcolor[rgb]{0.53,0.53,0.53}{##1}}}
\expandafter\def\csname PY@tok@gt\endcsname{\def\PY@tc##1{\textcolor[rgb]{0.00,0.27,0.87}{##1}}}
\expandafter\def\csname PY@tok@err\endcsname{\def\PY@bc##1{\setlength{\fboxsep}{0pt}\fcolorbox[rgb]{1.00,0.00,0.00}{1,1,1}{\strut ##1}}}
\expandafter\def\csname PY@tok@kc\endcsname{\let\PY@bf=\textbf\def\PY@tc##1{\textcolor[rgb]{0.00,0.50,0.00}{##1}}}
\expandafter\def\csname PY@tok@kd\endcsname{\let\PY@bf=\textbf\def\PY@tc##1{\textcolor[rgb]{0.00,0.50,0.00}{##1}}}
\expandafter\def\csname PY@tok@kn\endcsname{\let\PY@bf=\textbf\def\PY@tc##1{\textcolor[rgb]{0.00,0.50,0.00}{##1}}}
\expandafter\def\csname PY@tok@kr\endcsname{\let\PY@bf=\textbf\def\PY@tc##1{\textcolor[rgb]{0.00,0.50,0.00}{##1}}}
\expandafter\def\csname PY@tok@bp\endcsname{\def\PY@tc##1{\textcolor[rgb]{0.00,0.50,0.00}{##1}}}
\expandafter\def\csname PY@tok@fm\endcsname{\def\PY@tc##1{\textcolor[rgb]{0.00,0.00,1.00}{##1}}}
\expandafter\def\csname PY@tok@vc\endcsname{\def\PY@tc##1{\textcolor[rgb]{0.10,0.09,0.49}{##1}}}
\expandafter\def\csname PY@tok@vg\endcsname{\def\PY@tc##1{\textcolor[rgb]{0.10,0.09,0.49}{##1}}}
\expandafter\def\csname PY@tok@vi\endcsname{\def\PY@tc##1{\textcolor[rgb]{0.10,0.09,0.49}{##1}}}
\expandafter\def\csname PY@tok@vm\endcsname{\def\PY@tc##1{\textcolor[rgb]{0.10,0.09,0.49}{##1}}}
\expandafter\def\csname PY@tok@sa\endcsname{\def\PY@tc##1{\textcolor[rgb]{0.73,0.13,0.13}{##1}}}
\expandafter\def\csname PY@tok@sb\endcsname{\def\PY@tc##1{\textcolor[rgb]{0.73,0.13,0.13}{##1}}}
\expandafter\def\csname PY@tok@sc\endcsname{\def\PY@tc##1{\textcolor[rgb]{0.73,0.13,0.13}{##1}}}
\expandafter\def\csname PY@tok@dl\endcsname{\def\PY@tc##1{\textcolor[rgb]{0.73,0.13,0.13}{##1}}}
\expandafter\def\csname PY@tok@s2\endcsname{\def\PY@tc##1{\textcolor[rgb]{0.73,0.13,0.13}{##1}}}
\expandafter\def\csname PY@tok@sh\endcsname{\def\PY@tc##1{\textcolor[rgb]{0.73,0.13,0.13}{##1}}}
\expandafter\def\csname PY@tok@s1\endcsname{\def\PY@tc##1{\textcolor[rgb]{0.73,0.13,0.13}{##1}}}
\expandafter\def\csname PY@tok@mb\endcsname{\def\PY@tc##1{\textcolor[rgb]{0.40,0.40,0.40}{##1}}}
\expandafter\def\csname PY@tok@mf\endcsname{\def\PY@tc##1{\textcolor[rgb]{0.40,0.40,0.40}{##1}}}
\expandafter\def\csname PY@tok@mh\endcsname{\def\PY@tc##1{\textcolor[rgb]{0.40,0.40,0.40}{##1}}}
\expandafter\def\csname PY@tok@mi\endcsname{\def\PY@tc##1{\textcolor[rgb]{0.40,0.40,0.40}{##1}}}
\expandafter\def\csname PY@tok@il\endcsname{\def\PY@tc##1{\textcolor[rgb]{0.40,0.40,0.40}{##1}}}
\expandafter\def\csname PY@tok@mo\endcsname{\def\PY@tc##1{\textcolor[rgb]{0.40,0.40,0.40}{##1}}}
\expandafter\def\csname PY@tok@ch\endcsname{\let\PY@it=\textit\def\PY@tc##1{\textcolor[rgb]{0.25,0.50,0.50}{##1}}}
\expandafter\def\csname PY@tok@cm\endcsname{\let\PY@it=\textit\def\PY@tc##1{\textcolor[rgb]{0.25,0.50,0.50}{##1}}}
\expandafter\def\csname PY@tok@cpf\endcsname{\let\PY@it=\textit\def\PY@tc##1{\textcolor[rgb]{0.25,0.50,0.50}{##1}}}
\expandafter\def\csname PY@tok@c1\endcsname{\let\PY@it=\textit\def\PY@tc##1{\textcolor[rgb]{0.25,0.50,0.50}{##1}}}
\expandafter\def\csname PY@tok@cs\endcsname{\let\PY@it=\textit\def\PY@tc##1{\textcolor[rgb]{0.25,0.50,0.50}{##1}}}

\def\PYZbs{\char`\\}
\def\PYZus{\char`\_}
\def\PYZob{\char`\{}
\def\PYZcb{\char`\}}
\def\PYZca{\char`\^}
\def\PYZam{\char`\&}
\def\PYZlt{\char`\<}
\def\PYZgt{\char`\>}
\def\PYZsh{\char`\#}
\def\PYZpc{\char`\%}
\def\PYZdl{\char`\$}
\def\PYZhy{\char`\-}
\def\PYZsq{\char`\'}
\def\PYZdq{\char`\"}
\def\PYZti{\char`\~}
% for compatibility with earlier versions
\def\PYZat{@}
\def\PYZlb{[}
\def\PYZrb{]}
\makeatother


    % Exact colors from NB
    \definecolor{incolor}{rgb}{0.0, 0.0, 0.5}
    \definecolor{outcolor}{rgb}{0.545, 0.0, 0.0}



    
    % Prevent overflowing lines due to hard-to-break entities
    \sloppy 
    % Setup hyperref package
    \hypersetup{
      breaklinks=true,  % so long urls are correctly broken across lines
      colorlinks=true,
      urlcolor=urlcolor,
      linkcolor=linkcolor,
      citecolor=citecolor,
      }
    % Slightly bigger margins than the latex defaults
    
    \geometry{verbose,tmargin=1in,bmargin=1in,lmargin=1in,rmargin=1in}
    
    

    \begin{document}
    
    
    \maketitle
    
    

    
    \begin{Verbatim}[commandchars=\\\{\}]
{\color{incolor}In [{\color{incolor}1}]:} \PY{k+kn}{import} \PY{n+nn}{pandas} \PY{k}{as} \PY{n+nn}{pd}
        \PY{k+kn}{import} \PY{n+nn}{seaborn} \PY{k}{as} \PY{n+nn}{sns}
        \PY{k+kn}{import} \PY{n+nn}{matplotlib}\PY{n+nn}{.}\PY{n+nn}{pyplot} \PY{k}{as} \PY{n+nn}{plt}
        \PY{k+kn}{import} \PY{n+nn}{numpy} \PY{k}{as} \PY{n+nn}{np}
        
        \PY{k+kn}{import} \PY{n+nn}{warnings}\PY{p}{;} \PY{n}{warnings}\PY{o}{.}\PY{n}{simplefilter}\PY{p}{(}\PY{l+s+s1}{\PYZsq{}}\PY{l+s+s1}{ignore}\PY{l+s+s1}{\PYZsq{}}\PY{p}{)}
\end{Verbatim}


    \begin{Verbatim}[commandchars=\\\{\}]
{\color{incolor}In [{\color{incolor}2}]:} \PY{n}{datasets\PYZus{}path} \PY{o}{=} \PY{l+s+s1}{\PYZsq{}}\PY{l+s+s1}{/home/monodeepdas112/container\PYZhy{}vrts/AppliedAICourse/Datasets/}\PY{l+s+s1}{\PYZsq{}}
        \PY{n}{dataset} \PY{o}{=} \PY{l+s+s1}{\PYZsq{}}\PY{l+s+s1}{haberman.csv}\PY{l+s+s1}{\PYZsq{}}
        \PY{n}{haberman} \PY{o}{=} \PY{n}{pd}\PY{o}{.}\PY{n}{read\PYZus{}csv}\PY{p}{(}\PY{n}{datasets\PYZus{}path}\PY{o}{+}\PY{n}{dataset}\PY{p}{)}
\end{Verbatim}


    \subsection{Description of the
dataset}\label{description-of-the-dataset}

    \begin{Verbatim}[commandchars=\\\{\}]
{\color{incolor}In [{\color{incolor}3}]:} \PY{c+c1}{\PYZsh{}Converting to proper dataset column names}
        \PY{n}{haberman}\PY{o}{.}\PY{n}{columns} \PY{o}{=} \PY{n}{pd}\PY{o}{.}\PY{n}{Series}\PY{p}{(}\PY{p}{[}\PY{l+s+s1}{\PYZsq{}}\PY{l+s+s1}{Age}\PY{l+s+s1}{\PYZsq{}}\PY{p}{,} \PY{l+s+s1}{\PYZsq{}}\PY{l+s+s1}{Op\PYZus{}Year}\PY{l+s+s1}{\PYZsq{}}\PY{p}{,} \PY{l+s+s1}{\PYZsq{}}\PY{l+s+s1}{Pos\PYZus{}Ax\PYZus{}nodes}\PY{l+s+s1}{\PYZsq{}}\PY{p}{,} \PY{l+s+s1}{\PYZsq{}}\PY{l+s+s1}{Survival\PYZus{}Status}\PY{l+s+s1}{\PYZsq{}}\PY{p}{]}\PY{p}{)}
\end{Verbatim}


    \begin{Verbatim}[commandchars=\\\{\}]
{\color{incolor}In [{\color{incolor}9}]:} \PY{c+c1}{\PYZsh{}Shape}
        \PY{n+nb}{print}\PY{p}{(}\PY{l+s+s1}{\PYZsq{}}\PY{l+s+s1}{Dataset Shape : }\PY{l+s+s1}{\PYZsq{}}\PY{p}{,} \PY{n}{haberman}\PY{o}{.}\PY{n}{shape}\PY{p}{)}
        
        \PY{c+c1}{\PYZsh{}Columns in the dataset}
        \PY{n+nb}{print}\PY{p}{(}\PY{l+s+s1}{\PYZsq{}}\PY{l+s+se}{\PYZbs{}n}\PY{l+s+se}{\PYZbs{}n}\PY{l+s+s1}{Columns in the dataset : }\PY{l+s+s1}{\PYZsq{}}\PY{p}{,} \PY{n}{haberman}\PY{o}{.}\PY{n}{columns}\PY{p}{)}
\end{Verbatim}


    \begin{Verbatim}[commandchars=\\\{\}]
Dataset Shape :  (305, 4)


Columns in the dataset :  Index(['Age', 'Op\_Year', 'Pos\_Ax\_nodes', 'Survival\_Status'], dtype='object')

    \end{Verbatim}

    \begin{Verbatim}[commandchars=\\\{\}]
{\color{incolor}In [{\color{incolor}10}]:} \PY{c+c1}{\PYZsh{}Information about the dataset}
         \PY{n+nb}{print}\PY{p}{(}\PY{l+s+s1}{\PYZsq{}}\PY{l+s+se}{\PYZbs{}n}\PY{l+s+se}{\PYZbs{}n}\PY{l+s+s1}{Information about the dataset : }\PY{l+s+s1}{\PYZsq{}}\PY{p}{,} \PY{n}{haberman}\PY{o}{.}\PY{n}{info}\PY{p}{(}\PY{p}{)}\PY{p}{)}
         
         \PY{c+c1}{\PYZsh{}Description of dataset}
         \PY{n+nb}{print}\PY{p}{(}\PY{l+s+s1}{\PYZsq{}}\PY{l+s+se}{\PYZbs{}n}\PY{l+s+s1}{Description :}\PY{l+s+se}{\PYZbs{}n}\PY{l+s+s1}{\PYZsq{}}\PY{p}{,} \PY{n}{haberman}\PY{o}{.}\PY{n}{describe}\PY{p}{(}\PY{p}{)}\PY{p}{)}
\end{Verbatim}


    \begin{Verbatim}[commandchars=\\\{\}]
<class 'pandas.core.frame.DataFrame'>
RangeIndex: 305 entries, 0 to 304
Data columns (total 4 columns):
Age                305 non-null int64
Op\_Year            305 non-null int64
Pos\_Ax\_nodes       305 non-null int64
Survival\_Status    305 non-null int64
dtypes: int64(4)
memory usage: 9.6 KB


Information about the dataset :  None

Description :
               Age     Op\_Year  Pos\_Ax\_nodes  Survival\_Status
count  305.000000  305.000000    305.000000       305.000000
mean    52.531148   62.849180      4.036066         1.265574
std     10.744024    3.254078      7.199370         0.442364
min     30.000000   58.000000      0.000000         1.000000
25\%     44.000000   60.000000      0.000000         1.000000
50\%     52.000000   63.000000      1.000000         1.000000
75\%     61.000000   66.000000      4.000000         2.000000
max     83.000000   69.000000     52.000000         2.000000

    \end{Verbatim}

    As we can see there are no nulls present in the database

    \subsubsection{Conclusions from dataset
description}\label{conclusions-from-dataset-description}

    \paragraph{1. Total number of samples is 305 without any null
values}\label{total-number-of-samples-is-305-without-any-null-values}

\paragraph{2. There are 4 Attributes Age, Operation Year, Number of Axil
nodes found, Surival
Status}\label{there-are-4-attributes-age-operation-year-number-of-axil-nodes-found-surival-status}

\paragraph{3. Relevant statistical indormation can be found out in the
cell
above}\label{relevant-statistical-indormation-can-be-found-out-in-the-cell-above}

    \subsubsection{Scatter Plot
Visualization}\label{scatter-plot-visualization}

    \begin{Verbatim}[commandchars=\\\{\}]
{\color{incolor}In [{\color{incolor}16}]:} \PY{n}{sns}\PY{o}{.}\PY{n}{set\PYZus{}style}\PY{p}{(}\PY{l+s+s2}{\PYZdq{}}\PY{l+s+s2}{whitegrid}\PY{l+s+s2}{\PYZdq{}}\PY{p}{)}\PY{p}{;}
         \PY{n}{sns}\PY{o}{.}\PY{n}{FacetGrid}\PY{p}{(}\PY{n}{haberman}\PY{p}{,} \PY{n}{hue}\PY{o}{=}\PY{l+s+s2}{\PYZdq{}}\PY{l+s+s2}{Survival\PYZus{}Status}\PY{l+s+s2}{\PYZdq{}}\PY{p}{,} \PY{n}{size}\PY{o}{=}\PY{l+m+mi}{4}\PY{p}{)} \PYZbs{}
            \PY{o}{.}\PY{n}{map}\PY{p}{(}\PY{n}{plt}\PY{o}{.}\PY{n}{scatter}\PY{p}{,} \PY{l+s+s2}{\PYZdq{}}\PY{l+s+s2}{Age}\PY{l+s+s2}{\PYZdq{}}\PY{p}{,} \PY{l+s+s2}{\PYZdq{}}\PY{l+s+s2}{Pos\PYZus{}Ax\PYZus{}nodes}\PY{l+s+s2}{\PYZdq{}}\PY{p}{)} \PYZbs{}
            \PY{o}{.}\PY{n}{add\PYZus{}legend}\PY{p}{(}\PY{p}{)}\PY{p}{;}
         \PY{n}{plt}\PY{o}{.}\PY{n}{show}\PY{p}{(}\PY{p}{)}\PY{p}{;}
\end{Verbatim}


    \begin{center}
    \adjustimage{max size={0.9\linewidth}{0.9\paperheight}}{output_10_0.png}
    \end{center}
    { \hspace*{\fill} \\}
    
    \paragraph{Not much sense can be made out of this plot as the scatter
plot seems to show no particular pattern so we can say that there exists
no relation between age of the patient and the number of positive axill
nodes
found.}\label{not-much-sense-can-be-made-out-of-this-plot-as-the-scatter-plot-seems-to-show-no-particular-pattern-so-we-can-say-that-there-exists-no-relation-between-age-of-the-patient-and-the-number-of-positive-axill-nodes-found.}

    \begin{Verbatim}[commandchars=\\\{\}]
{\color{incolor}In [{\color{incolor}17}]:} \PY{n}{sns}\PY{o}{.}\PY{n}{set\PYZus{}style}\PY{p}{(}\PY{l+s+s2}{\PYZdq{}}\PY{l+s+s2}{whitegrid}\PY{l+s+s2}{\PYZdq{}}\PY{p}{)}\PY{p}{;}
         \PY{n}{sns}\PY{o}{.}\PY{n}{FacetGrid}\PY{p}{(}\PY{n}{haberman}\PY{p}{,} \PY{n}{hue}\PY{o}{=}\PY{l+s+s2}{\PYZdq{}}\PY{l+s+s2}{Survival\PYZus{}Status}\PY{l+s+s2}{\PYZdq{}}\PY{p}{,} \PY{n}{size}\PY{o}{=}\PY{l+m+mi}{4}\PY{p}{)} \PYZbs{}
            \PY{o}{.}\PY{n}{map}\PY{p}{(}\PY{n}{plt}\PY{o}{.}\PY{n}{scatter}\PY{p}{,} \PY{l+s+s2}{\PYZdq{}}\PY{l+s+s2}{Op\PYZus{}Year}\PY{l+s+s2}{\PYZdq{}}\PY{p}{,} \PY{l+s+s2}{\PYZdq{}}\PY{l+s+s2}{Pos\PYZus{}Ax\PYZus{}nodes}\PY{l+s+s2}{\PYZdq{}}\PY{p}{)} \PYZbs{}
            \PY{o}{.}\PY{n}{add\PYZus{}legend}\PY{p}{(}\PY{p}{)}\PY{p}{;}
         \PY{n}{plt}\PY{o}{.}\PY{n}{show}\PY{p}{(}\PY{p}{)}\PY{p}{;}
\end{Verbatim}


    \begin{center}
    \adjustimage{max size={0.9\linewidth}{0.9\paperheight}}{output_12_0.png}
    \end{center}
    { \hspace*{\fill} \\}
    
    \paragraph{Even this plot is almost evenly spread with no particular
pattern}\label{even-this-plot-is-almost-evenly-spread-with-no-particular-pattern}

    \begin{Verbatim}[commandchars=\\\{\}]
{\color{incolor}In [{\color{incolor}18}]:} \PY{n}{sns}\PY{o}{.}\PY{n}{set\PYZus{}style}\PY{p}{(}\PY{l+s+s2}{\PYZdq{}}\PY{l+s+s2}{whitegrid}\PY{l+s+s2}{\PYZdq{}}\PY{p}{)}\PY{p}{;}
         \PY{n}{sns}\PY{o}{.}\PY{n}{FacetGrid}\PY{p}{(}\PY{n}{haberman}\PY{p}{,} \PY{n}{hue}\PY{o}{=}\PY{l+s+s2}{\PYZdq{}}\PY{l+s+s2}{Survival\PYZus{}Status}\PY{l+s+s2}{\PYZdq{}}\PY{p}{,} \PY{n}{size}\PY{o}{=}\PY{l+m+mi}{4}\PY{p}{)} \PYZbs{}
            \PY{o}{.}\PY{n}{map}\PY{p}{(}\PY{n}{plt}\PY{o}{.}\PY{n}{scatter}\PY{p}{,} \PY{l+s+s2}{\PYZdq{}}\PY{l+s+s2}{Age}\PY{l+s+s2}{\PYZdq{}}\PY{p}{,} \PY{l+s+s2}{\PYZdq{}}\PY{l+s+s2}{Survival\PYZus{}Status}\PY{l+s+s2}{\PYZdq{}}\PY{p}{)} \PYZbs{}
            \PY{o}{.}\PY{n}{add\PYZus{}legend}\PY{p}{(}\PY{p}{)}\PY{p}{;}
         \PY{n}{plt}\PY{o}{.}\PY{n}{show}\PY{p}{(}\PY{p}{)}\PY{p}{;}
\end{Verbatim}


    \begin{center}
    \adjustimage{max size={0.9\linewidth}{0.9\paperheight}}{output_14_0.png}
    \end{center}
    { \hspace*{\fill} \\}
    
    \paragraph{Same as we cannot form any kind of inference from the plot as
both groups of patients share the common a large common range of
age}\label{same-as-we-cannot-form-any-kind-of-inference-from-the-plot-as-both-groups-of-patients-share-the-common-a-large-common-range-of-age}

    \begin{Verbatim}[commandchars=\\\{\}]
{\color{incolor}In [{\color{incolor}19}]:} \PY{n}{sns}\PY{o}{.}\PY{n}{set\PYZus{}style}\PY{p}{(}\PY{l+s+s2}{\PYZdq{}}\PY{l+s+s2}{whitegrid}\PY{l+s+s2}{\PYZdq{}}\PY{p}{)}\PY{p}{;}
         \PY{n}{sns}\PY{o}{.}\PY{n}{FacetGrid}\PY{p}{(}\PY{n}{haberman}\PY{p}{,} \PY{n}{hue}\PY{o}{=}\PY{l+s+s2}{\PYZdq{}}\PY{l+s+s2}{Survival\PYZus{}Status}\PY{l+s+s2}{\PYZdq{}}\PY{p}{,} \PY{n}{size}\PY{o}{=}\PY{l+m+mi}{4}\PY{p}{)} \PYZbs{}
            \PY{o}{.}\PY{n}{map}\PY{p}{(}\PY{n}{plt}\PY{o}{.}\PY{n}{scatter}\PY{p}{,} \PY{l+s+s2}{\PYZdq{}}\PY{l+s+s2}{Pos\PYZus{}Ax\PYZus{}nodes}\PY{l+s+s2}{\PYZdq{}}\PY{p}{,} \PY{l+s+s2}{\PYZdq{}}\PY{l+s+s2}{Survival\PYZus{}Status}\PY{l+s+s2}{\PYZdq{}}\PY{p}{)} \PYZbs{}
            \PY{o}{.}\PY{n}{add\PYZus{}legend}\PY{p}{(}\PY{p}{)}\PY{p}{;}
         \PY{n}{plt}\PY{o}{.}\PY{n}{show}\PY{p}{(}\PY{p}{)}\PY{p}{;}
\end{Verbatim}


    \begin{center}
    \adjustimage{max size={0.9\linewidth}{0.9\paperheight}}{output_16_0.png}
    \end{center}
    { \hspace*{\fill} \\}
    
    \paragraph{Scatter plot does show a density of class 1 nearer to 0 where
are the data points for class 2 seem more widely spread but not much of
inference can be drawn as we still see a large common range of
Pos\_Ax\_nodes which are occupied by both
classes}\label{scatter-plot-does-show-a-density-of-class-1-nearer-to-0-where-are-the-data-points-for-class-2-seem-more-widely-spread-but-not-much-of-inference-can-be-drawn-as-we-still-see-a-large-common-range-of-pos_ax_nodes-which-are-occupied-by-both-classes}

    \subsubsection{Pair-Plot}\label{pair-plot}

    \begin{Verbatim}[commandchars=\\\{\}]
{\color{incolor}In [{\color{incolor}21}]:} \PY{n}{plt}\PY{o}{.}\PY{n}{close}\PY{p}{(}\PY{p}{)}\PY{p}{;}
         \PY{n}{sns}\PY{o}{.}\PY{n}{set\PYZus{}style}\PY{p}{(}\PY{l+s+s2}{\PYZdq{}}\PY{l+s+s2}{whitegrid}\PY{l+s+s2}{\PYZdq{}}\PY{p}{)}\PY{p}{;}
         \PY{n}{sns}\PY{o}{.}\PY{n}{pairplot}\PY{p}{(}\PY{n}{haberman}\PY{p}{,} \PY{n}{hue}\PY{o}{=}\PY{l+s+s2}{\PYZdq{}}\PY{l+s+s2}{Survival\PYZus{}Status}\PY{l+s+s2}{\PYZdq{}}\PY{p}{,} \PY{n}{size}\PY{o}{=}\PY{l+m+mi}{3}\PY{p}{)}\PY{p}{;}
         \PY{n}{plt}\PY{o}{.}\PY{n}{show}\PY{p}{(}\PY{p}{)}
\end{Verbatim}


    \begin{center}
    \adjustimage{max size={0.9\linewidth}{0.9\paperheight}}{output_19_0.png}
    \end{center}
    { \hspace*{\fill} \\}
    
    \subsection{PDF, CDF Analysis}\label{pdf-cdf-analysis}

    \begin{Verbatim}[commandchars=\\\{\}]
{\color{incolor}In [{\color{incolor}22}]:} \PY{n}{sns}\PY{o}{.}\PY{n}{FacetGrid}\PY{p}{(}\PY{n}{haberman}\PY{p}{,} \PY{n}{hue}\PY{o}{=}\PY{l+s+s2}{\PYZdq{}}\PY{l+s+s2}{Survival\PYZus{}Status}\PY{l+s+s2}{\PYZdq{}}\PY{p}{,} \PY{n}{size}\PY{o}{=}\PY{l+m+mi}{5}\PY{p}{)} \PYZbs{}
            \PY{o}{.}\PY{n}{map}\PY{p}{(}\PY{n}{sns}\PY{o}{.}\PY{n}{distplot}\PY{p}{,} \PY{l+s+s2}{\PYZdq{}}\PY{l+s+s2}{Pos\PYZus{}Ax\PYZus{}nodes}\PY{l+s+s2}{\PYZdq{}}\PY{p}{)} \PYZbs{}
            \PY{o}{.}\PY{n}{add\PYZus{}legend}\PY{p}{(}\PY{p}{)}\PY{p}{;}
         \PY{n}{plt}\PY{o}{.}\PY{n}{show}\PY{p}{(}\PY{p}{)}\PY{p}{;}
\end{Verbatim}


    \begin{center}
    \adjustimage{max size={0.9\linewidth}{0.9\paperheight}}{output_21_0.png}
    \end{center}
    { \hspace*{\fill} \\}
    
    \begin{Verbatim}[commandchars=\\\{\}]
{\color{incolor}In [{\color{incolor}23}]:} \PY{n}{sns}\PY{o}{.}\PY{n}{FacetGrid}\PY{p}{(}\PY{n}{haberman}\PY{p}{,} \PY{n}{hue}\PY{o}{=}\PY{l+s+s2}{\PYZdq{}}\PY{l+s+s2}{Survival\PYZus{}Status}\PY{l+s+s2}{\PYZdq{}}\PY{p}{,} \PY{n}{size}\PY{o}{=}\PY{l+m+mi}{5}\PY{p}{)} \PYZbs{}
            \PY{o}{.}\PY{n}{map}\PY{p}{(}\PY{n}{sns}\PY{o}{.}\PY{n}{distplot}\PY{p}{,} \PY{l+s+s2}{\PYZdq{}}\PY{l+s+s2}{Age}\PY{l+s+s2}{\PYZdq{}}\PY{p}{)} \PYZbs{}
            \PY{o}{.}\PY{n}{add\PYZus{}legend}\PY{p}{(}\PY{p}{)}\PY{p}{;}
         \PY{n}{plt}\PY{o}{.}\PY{n}{show}\PY{p}{(}\PY{p}{)}\PY{p}{;}
\end{Verbatim}


    \begin{center}
    \adjustimage{max size={0.9\linewidth}{0.9\paperheight}}{output_22_0.png}
    \end{center}
    { \hspace*{\fill} \\}
    
    \begin{Verbatim}[commandchars=\\\{\}]
{\color{incolor}In [{\color{incolor}24}]:} \PY{n}{sns}\PY{o}{.}\PY{n}{FacetGrid}\PY{p}{(}\PY{n}{haberman}\PY{p}{,} \PY{n}{hue}\PY{o}{=}\PY{l+s+s2}{\PYZdq{}}\PY{l+s+s2}{Survival\PYZus{}Status}\PY{l+s+s2}{\PYZdq{}}\PY{p}{,} \PY{n}{size}\PY{o}{=}\PY{l+m+mi}{5}\PY{p}{)} \PYZbs{}
            \PY{o}{.}\PY{n}{map}\PY{p}{(}\PY{n}{sns}\PY{o}{.}\PY{n}{distplot}\PY{p}{,} \PY{l+s+s2}{\PYZdq{}}\PY{l+s+s2}{Op\PYZus{}Year}\PY{l+s+s2}{\PYZdq{}}\PY{p}{)} \PYZbs{}
            \PY{o}{.}\PY{n}{add\PYZus{}legend}\PY{p}{(}\PY{p}{)}\PY{p}{;}
         \PY{n}{plt}\PY{o}{.}\PY{n}{show}\PY{p}{(}\PY{p}{)}\PY{p}{;}
\end{Verbatim}


    \begin{center}
    \adjustimage{max size={0.9\linewidth}{0.9\paperheight}}{output_23_0.png}
    \end{center}
    { \hspace*{\fill} \\}
    
    \begin{Verbatim}[commandchars=\\\{\}]
{\color{incolor}In [{\color{incolor}54}]:} \PY{n}{counts}\PY{p}{,} \PY{n}{bin\PYZus{}edges} \PY{o}{=} \PY{n}{np}\PY{o}{.}\PY{n}{histogram}\PY{p}{(}\PY{n}{haberman}\PY{p}{[}\PY{l+s+s1}{\PYZsq{}}\PY{l+s+s1}{Pos\PYZus{}Ax\PYZus{}nodes}\PY{l+s+s1}{\PYZsq{}}\PY{p}{]}\PY{p}{,} \PY{n}{bins}\PY{o}{=}\PY{l+m+mi}{10}\PY{p}{,} 
                                          \PY{n}{density} \PY{o}{=} \PY{k+kc}{True}\PY{p}{)}
         \PY{n}{pdf} \PY{o}{=} \PY{n}{counts}\PY{o}{/}\PY{p}{(}\PY{n+nb}{sum}\PY{p}{(}\PY{n}{counts}\PY{p}{)}\PY{p}{)}
         \PY{n+nb}{print}\PY{p}{(}\PY{n}{pdf}\PY{p}{)}\PY{p}{;}
         \PY{n+nb}{print}\PY{p}{(}\PY{n}{bin\PYZus{}edges}\PY{p}{)}\PY{p}{;}
         \PY{n}{cdf} \PY{o}{=} \PY{n}{np}\PY{o}{.}\PY{n}{cumsum}\PY{p}{(}\PY{n}{pdf}\PY{p}{)}
         \PY{n}{plt}\PY{o}{.}\PY{n}{plot}\PY{p}{(}\PY{n}{bin\PYZus{}edges}\PY{p}{[}\PY{l+m+mi}{1}\PY{p}{:}\PY{p}{]}\PY{p}{,}\PY{n}{pdf}\PY{p}{)}\PY{p}{;}
         \PY{n}{plt}\PY{o}{.}\PY{n}{plot}\PY{p}{(}\PY{n}{bin\PYZus{}edges}\PY{p}{[}\PY{l+m+mi}{1}\PY{p}{:}\PY{p}{]}\PY{p}{,} \PY{n}{cdf}\PY{p}{)}
         \PY{n}{plt}\PY{o}{.}\PY{n}{xlabel}\PY{p}{(}\PY{l+s+s1}{\PYZsq{}}\PY{l+s+s1}{Pos\PYZus{}Ax\PYZus{}nodes}\PY{l+s+s1}{\PYZsq{}}\PY{p}{)}
         
         \PY{n}{plt}\PY{o}{.}\PY{n}{show}\PY{p}{(}\PY{p}{)}\PY{p}{;}
\end{Verbatim}


    \begin{Verbatim}[commandchars=\\\{\}]
[0.7704918  0.09836066 0.05901639 0.02622951 0.0295082  0.00655738
 0.00327869 0.         0.00327869 0.00327869]
[ 0.   5.2 10.4 15.6 20.8 26.  31.2 36.4 41.6 46.8 52. ]

    \end{Verbatim}

    \begin{center}
    \adjustimage{max size={0.9\linewidth}{0.9\paperheight}}{output_24_1.png}
    \end{center}
    { \hspace*{\fill} \\}
    
    \begin{Verbatim}[commandchars=\\\{\}]
{\color{incolor}In [{\color{incolor}55}]:} \PY{n}{counts}\PY{p}{,} \PY{n}{bin\PYZus{}edges} \PY{o}{=} \PY{n}{np}\PY{o}{.}\PY{n}{histogram}\PY{p}{(}\PY{n}{haberman}\PY{p}{[}\PY{l+s+s1}{\PYZsq{}}\PY{l+s+s1}{Age}\PY{l+s+s1}{\PYZsq{}}\PY{p}{]}\PY{p}{,} \PY{n}{bins}\PY{o}{=}\PY{l+m+mi}{10}\PY{p}{,} 
                                          \PY{n}{density} \PY{o}{=} \PY{k+kc}{True}\PY{p}{)}
         \PY{n}{pdf} \PY{o}{=} \PY{n}{counts}\PY{o}{/}\PY{p}{(}\PY{n+nb}{sum}\PY{p}{(}\PY{n}{counts}\PY{p}{)}\PY{p}{)}
         \PY{n+nb}{print}\PY{p}{(}\PY{n}{pdf}\PY{p}{)}\PY{p}{;}
         \PY{n+nb}{print}\PY{p}{(}\PY{n}{bin\PYZus{}edges}\PY{p}{)}\PY{p}{;}
         \PY{n}{cdf} \PY{o}{=} \PY{n}{np}\PY{o}{.}\PY{n}{cumsum}\PY{p}{(}\PY{n}{pdf}\PY{p}{)}
         \PY{n}{plt}\PY{o}{.}\PY{n}{plot}\PY{p}{(}\PY{n}{bin\PYZus{}edges}\PY{p}{[}\PY{l+m+mi}{1}\PY{p}{:}\PY{p}{]}\PY{p}{,}\PY{n}{pdf}\PY{p}{)}\PY{p}{;}
         \PY{n}{plt}\PY{o}{.}\PY{n}{plot}\PY{p}{(}\PY{n}{bin\PYZus{}edges}\PY{p}{[}\PY{l+m+mi}{1}\PY{p}{:}\PY{p}{]}\PY{p}{,} \PY{n}{cdf}\PY{p}{)}
         \PY{n}{plt}\PY{o}{.}\PY{n}{xlabel}\PY{p}{(}\PY{l+s+s1}{\PYZsq{}}\PY{l+s+s1}{Age}\PY{l+s+s1}{\PYZsq{}}\PY{p}{)}
         \PY{n}{plt}\PY{o}{.}\PY{n}{show}\PY{p}{(}\PY{p}{)}\PY{p}{;}
\end{Verbatim}


    \begin{Verbatim}[commandchars=\\\{\}]
[0.04918033 0.08852459 0.15081967 0.17377049 0.18032787 0.13442623
 0.13442623 0.05901639 0.02295082 0.00655738]
[30.  35.3 40.6 45.9 51.2 56.5 61.8 67.1 72.4 77.7 83. ]

    \end{Verbatim}

    \begin{center}
    \adjustimage{max size={0.9\linewidth}{0.9\paperheight}}{output_25_1.png}
    \end{center}
    { \hspace*{\fill} \\}
    
    \begin{Verbatim}[commandchars=\\\{\}]
{\color{incolor}In [{\color{incolor}56}]:} \PY{n}{counts}\PY{p}{,} \PY{n}{bin\PYZus{}edges} \PY{o}{=} \PY{n}{np}\PY{o}{.}\PY{n}{histogram}\PY{p}{(}\PY{n}{haberman}\PY{p}{[}\PY{l+s+s1}{\PYZsq{}}\PY{l+s+s1}{Op\PYZus{}Year}\PY{l+s+s1}{\PYZsq{}}\PY{p}{]}\PY{p}{,} \PY{n}{bins}\PY{o}{=}\PY{l+m+mi}{10}\PY{p}{,} 
                                          \PY{n}{density} \PY{o}{=} \PY{k+kc}{True}\PY{p}{)}
         \PY{n}{pdf} \PY{o}{=} \PY{n}{counts}\PY{o}{/}\PY{p}{(}\PY{n+nb}{sum}\PY{p}{(}\PY{n}{counts}\PY{p}{)}\PY{p}{)}
         \PY{n+nb}{print}\PY{p}{(}\PY{n}{pdf}\PY{p}{)}\PY{p}{;}
         \PY{n+nb}{print}\PY{p}{(}\PY{n}{bin\PYZus{}edges}\PY{p}{)}\PY{p}{;}
         \PY{n}{cdf} \PY{o}{=} \PY{n}{np}\PY{o}{.}\PY{n}{cumsum}\PY{p}{(}\PY{n}{pdf}\PY{p}{)}
         \PY{n}{plt}\PY{o}{.}\PY{n}{plot}\PY{p}{(}\PY{n}{bin\PYZus{}edges}\PY{p}{[}\PY{l+m+mi}{1}\PY{p}{:}\PY{p}{]}\PY{p}{,}\PY{n}{pdf}\PY{p}{)}\PY{p}{;}
         \PY{n}{plt}\PY{o}{.}\PY{n}{plot}\PY{p}{(}\PY{n}{bin\PYZus{}edges}\PY{p}{[}\PY{l+m+mi}{1}\PY{p}{:}\PY{p}{]}\PY{p}{,} \PY{n}{cdf}\PY{p}{)}
         \PY{n}{plt}\PY{o}{.}\PY{n}{xlabel}\PY{p}{(}\PY{l+s+s1}{\PYZsq{}}\PY{l+s+s1}{Op\PYZus{}Year}\PY{l+s+s1}{\PYZsq{}}\PY{p}{)}
         \PY{n}{plt}\PY{o}{.}\PY{n}{show}\PY{p}{(}\PY{p}{)}\PY{p}{;}
\end{Verbatim}


    \begin{Verbatim}[commandchars=\\\{\}]
[0.20655738 0.09180328 0.0852459  0.07540984 0.09836066 0.09836066
 0.09180328 0.09180328 0.08196721 0.07868852]
[58.  59.1 60.2 61.3 62.4 63.5 64.6 65.7 66.8 67.9 69. ]

    \end{Verbatim}

    \begin{center}
    \adjustimage{max size={0.9\linewidth}{0.9\paperheight}}{output_26_1.png}
    \end{center}
    { \hspace*{\fill} \\}
    
    \paragraph{Q. How many people lived more than 5 years and how many less
than that
?}\label{q.-how-many-people-lived-more-than-5-years-and-how-many-less-than-that}

    \begin{Verbatim}[commandchars=\\\{\}]
{\color{incolor}In [{\color{incolor}43}]:} \PY{n}{sns}\PY{o}{.}\PY{n}{distplot}\PY{p}{(}\PY{n}{haberman}\PY{o}{.}\PY{n}{Survival\PYZus{}Status}\PY{p}{,} \PY{n}{kde}\PY{o}{=}\PY{k+kc}{False}\PY{p}{,} \PY{n}{rug}\PY{o}{=}\PY{k+kc}{True}\PY{p}{)}\PY{p}{;}
\end{Verbatim}


    \begin{center}
    \adjustimage{max size={0.9\linewidth}{0.9\paperheight}}{output_28_0.png}
    \end{center}
    { \hspace*{\fill} \\}
    
    \paragraph{Approx 225 people who lived more than 5 years after their
operation and approx 75 people who lived less than 5
years}\label{approx-225-people-who-lived-more-than-5-years-after-their-operation-and-approx-75-people-who-lived-less-than-5-years}

    \paragraph{Q. How many Number of Axil nodes were found for the people
who lived more than 5 years
?}\label{q.-how-many-number-of-axil-nodes-were-found-for-the-people-who-lived-more-than-5-years}

    \begin{Verbatim}[commandchars=\\\{\}]
{\color{incolor}In [{\color{incolor}44}]:} \PY{n}{long\PYZus{}lived}\PY{o}{=}\PY{n}{haberman}\PY{o}{.}\PY{n}{loc}\PY{p}{[}\PY{n}{haberman}\PY{p}{[}\PY{l+s+s1}{\PYZsq{}}\PY{l+s+s1}{Survival\PYZus{}Status}\PY{l+s+s1}{\PYZsq{}}\PY{p}{]} \PY{o}{==} \PY{l+m+mi}{1}\PY{p}{]}
         \PY{n+nb}{print}\PY{p}{(}\PY{l+s+s1}{\PYZsq{}}\PY{l+s+s1}{Number of people who lived for more than 5 years : }\PY{l+s+se}{\PYZbs{}n}\PY{l+s+se}{\PYZbs{}n}\PY{l+s+s1}{\PYZsq{}}\PY{p}{,} \PY{n}{long\PYZus{}lived}\PY{o}{.}\PY{n}{describe}\PY{p}{(}\PY{p}{)}\PY{p}{)}
         
         \PY{n}{short\PYZus{}lived}\PY{o}{=}\PY{n}{haberman}\PY{o}{.}\PY{n}{loc}\PY{p}{[}\PY{n}{haberman}\PY{p}{[}\PY{l+s+s1}{\PYZsq{}}\PY{l+s+s1}{Survival\PYZus{}Status}\PY{l+s+s1}{\PYZsq{}}\PY{p}{]}\PY{o}{==}\PY{l+m+mi}{2}\PY{p}{]}
         \PY{n+nb}{print}\PY{p}{(}\PY{l+s+s1}{\PYZsq{}}\PY{l+s+se}{\PYZbs{}n}\PY{l+s+se}{\PYZbs{}n}\PY{l+s+s1}{Number of people who lived for less than 5 years : }\PY{l+s+se}{\PYZbs{}n}\PY{l+s+s1}{\PYZsq{}}\PY{p}{,} \PY{n}{short\PYZus{}lived}\PY{o}{.}\PY{n}{describe}\PY{p}{(}\PY{p}{)}\PY{p}{)}
\end{Verbatim}


    \begin{Verbatim}[commandchars=\\\{\}]
Number of people who lived for more than 5 years : 

               Age     Op\_Year  Pos\_Ax\_nodes  Survival\_Status
count  224.000000  224.000000    224.000000            224.0
mean    52.116071   62.857143      2.799107              1.0
std     10.937446    3.229231      5.882237              0.0
min     30.000000   58.000000      0.000000              1.0
25\%     43.000000   60.000000      0.000000              1.0
50\%     52.000000   63.000000      0.000000              1.0
75\%     60.000000   66.000000      3.000000              1.0
max     77.000000   69.000000     46.000000              1.0


Number of people who lived for less than 5 years : 
              Age    Op\_Year  Pos\_Ax\_nodes  Survival\_Status
count  81.000000  81.000000     81.000000             81.0
mean   53.679012  62.827160      7.456790              2.0
std    10.167137   3.342118      9.185654              0.0
min    34.000000  58.000000      0.000000              2.0
25\%    46.000000  59.000000      1.000000              2.0
50\%    53.000000  63.000000      4.000000              2.0
75\%    61.000000  65.000000     11.000000              2.0
max    83.000000  69.000000     52.000000              2.0

    \end{Verbatim}

    \paragraph{Mean Number of Positive Axil Nodes : 7.456 for the people who
could not live for more than 5
years}\label{mean-number-of-positive-axil-nodes-7.456-for-the-people-who-could-not-live-for-more-than-5-years}

\paragraph{Mean Number of Positive Axil Nodes : 2.799 for the people who
could live for more than 5
years}\label{mean-number-of-positive-axil-nodes-2.799-for-the-people-who-could-live-for-more-than-5-years}

\paragraph{Gives us a clear picture of the spread of the disease and
thus (intuitively) can be used as a important reason for the early death
of the people as in their case the disease spread a lot
more.}\label{gives-us-a-clear-picture-of-the-spread-of-the-disease-and-thus-intuitively-can-be-used-as-a-important-reason-for-the-early-death-of-the-people-as-in-their-case-the-disease-spread-a-lot-more.}

    \begin{Verbatim}[commandchars=\\\{\}]
{\color{incolor}In [{\color{incolor}45}]:} \PY{c+c1}{\PYZsh{}Finding out the medians}
         \PY{n+nb}{print}\PY{p}{(}\PY{l+s+s1}{\PYZsq{}}\PY{l+s+s1}{Median Number of Positive Axil Nodes found for class 1 : }\PY{l+s+s1}{\PYZsq{}}\PY{p}{,} \PY{n}{np}\PY{o}{.}\PY{n}{median}\PY{p}{(}\PY{n}{long\PYZus{}lived}\PY{p}{)}\PY{p}{)}
         \PY{n+nb}{print}\PY{p}{(}\PY{l+s+s1}{\PYZsq{}}\PY{l+s+s1}{Median Number of Positive Axil Nodes found for class 2 : }\PY{l+s+s1}{\PYZsq{}}\PY{p}{,} \PY{n}{np}\PY{o}{.}\PY{n}{median}\PY{p}{(}\PY{n}{short\PYZus{}lived}\PY{p}{)}\PY{p}{)}
\end{Verbatim}


    \begin{Verbatim}[commandchars=\\\{\}]
Median Number of Positive Axil Nodes found for class 1 :  30.0
Median Number of Positive Axil Nodes found for class 2 :  34.5

    \end{Verbatim}

    \paragraph{Observations are confusing as the median of both the classes
differ only by 4 so it cannot be the spread of the disease that might
have been the crucial reason behind their deaths. Next Step trying out
Box and Violin plots so understand more about the density of the Number
of positive Axial Nodes
found}\label{observations-are-confusing-as-the-median-of-both-the-classes-differ-only-by-4-so-it-cannot-be-the-spread-of-the-disease-that-might-have-been-the-crucial-reason-behind-their-deaths.-next-step-trying-out-box-and-violin-plots-so-understand-more-about-the-density-of-the-number-of-positive-axial-nodes-found}

    \paragraph{Trying out box plot to see the information visually and to
determine the difference between the variance of the Num\_Ax\_nodes
between both
classes}\label{trying-out-box-plot-to-see-the-information-visually-and-to-determine-the-difference-between-the-variance-of-the-num_ax_nodes-between-both-classes}

    \begin{Verbatim}[commandchars=\\\{\}]
{\color{incolor}In [{\color{incolor}47}]:} \PY{n}{sns}\PY{o}{.}\PY{n}{boxplot}\PY{p}{(}\PY{n}{y}\PY{o}{=}\PY{l+s+s1}{\PYZsq{}}\PY{l+s+s1}{Pos\PYZus{}Ax\PYZus{}nodes}\PY{l+s+s1}{\PYZsq{}}\PY{p}{,} \PY{n}{x}\PY{o}{=}\PY{l+s+s1}{\PYZsq{}}\PY{l+s+s1}{Survival\PYZus{}Status}\PY{l+s+s1}{\PYZsq{}}\PY{p}{,}\PY{n}{data}\PY{o}{=}\PY{n}{haberman}\PY{p}{)}
         \PY{n}{plt}\PY{o}{.}\PY{n}{show}\PY{p}{(}\PY{p}{)}
\end{Verbatim}


    \begin{center}
    \adjustimage{max size={0.9\linewidth}{0.9\paperheight}}{output_36_0.png}
    \end{center}
    { \hspace*{\fill} \\}
    
    \begin{Verbatim}[commandchars=\\\{\}]
{\color{incolor}In [{\color{incolor}48}]:} \PY{c+c1}{\PYZsh{}Denser regions are darther and sparser regions are thinner}
         \PY{n}{sns}\PY{o}{.}\PY{n}{violinplot}\PY{p}{(}\PY{n}{y}\PY{o}{=}\PY{l+s+s1}{\PYZsq{}}\PY{l+s+s1}{Pos\PYZus{}Ax\PYZus{}nodes}\PY{l+s+s1}{\PYZsq{}}\PY{p}{,} \PY{n}{x}\PY{o}{=}\PY{l+s+s1}{\PYZsq{}}\PY{l+s+s1}{Survival\PYZus{}Status}\PY{l+s+s1}{\PYZsq{}}\PY{p}{,} \PY{n}{size}\PY{o}{=}\PY{l+m+mi}{10}\PY{p}{,} \PY{n}{data}\PY{o}{=}\PY{n}{haberman}\PY{p}{)}
         \PY{n}{plt}\PY{o}{.}\PY{n}{show}\PY{p}{(}\PY{p}{)}
\end{Verbatim}


    \begin{center}
    \adjustimage{max size={0.9\linewidth}{0.9\paperheight}}{output_37_0.png}
    \end{center}
    { \hspace*{\fill} \\}
    
    \paragraph{As is clearly evident from the data is that 75 percentile of
the people who could not live more than 5 years had 10 positive axial
nodes where are the 75 percentile of the people who could live longer
than 5 years is very much lesser (approx
\textless{}5)}\label{as-is-clearly-evident-from-the-data-is-that-75-percentile-of-the-people-who-could-not-live-more-than-5-years-had-10-positive-axial-nodes-where-are-the-75-percentile-of-the-people-who-could-live-longer-than-5-years-is-very-much-lesser-approx-5}

    \paragraph{As is also evident from the violin plots is that the median
is really very close to the 25 percentile in case 1 and the density of
people having lesser number of the positive nodes is higher and is below
10.}\label{as-is-also-evident-from-the-violin-plots-is-that-the-median-is-really-very-close-to-the-25-percentile-in-case-1-and-the-density-of-people-having-lesser-number-of-the-positive-nodes-is-higher-and-is-below-10.}

\paragraph{But there are more people with more positive nodes in class 2
which can be a reason why most of them could not live for more than 5
years.}\label{but-there-are-more-people-with-more-positive-nodes-in-class-2-which-can-be-a-reason-why-most-of-them-could-not-live-for-more-than-5-years.}

\paragraph{Although there are quite some outliers in class 1 who
survived to live more than 5 years despite having more number of
positive axial
nodes}\label{although-there-are-quite-some-outliers-in-class-1-who-survived-to-live-more-than-5-years-despite-having-more-number-of-positive-axial-nodes}

    \begin{Verbatim}[commandchars=\\\{\}]
{\color{incolor}In [{\color{incolor}49}]:} \PY{n}{sns}\PY{o}{.}\PY{n}{jointplot}\PY{p}{(}\PY{n}{y}\PY{o}{=}\PY{l+s+s1}{\PYZsq{}}\PY{l+s+s1}{Age}\PY{l+s+s1}{\PYZsq{}}\PY{p}{,} \PY{n}{x}\PY{o}{=}\PY{l+s+s1}{\PYZsq{}}\PY{l+s+s1}{Pos\PYZus{}Ax\PYZus{}nodes}\PY{l+s+s1}{\PYZsq{}}\PY{p}{,} \PY{n}{data}\PY{o}{=}\PY{n}{long\PYZus{}lived}\PY{p}{,} \PY{n}{kind}\PY{o}{=}\PY{l+s+s1}{\PYZsq{}}\PY{l+s+s1}{kde}\PY{l+s+s1}{\PYZsq{}}\PY{p}{)}
\end{Verbatim}


\begin{Verbatim}[commandchars=\\\{\}]
{\color{outcolor}Out[{\color{outcolor}49}]:} <seaborn.axisgrid.JointGrid at 0x7f81cf0e5d68>
\end{Verbatim}
            
    \begin{center}
    \adjustimage{max size={0.9\linewidth}{0.9\paperheight}}{output_40_1.png}
    \end{center}
    { \hspace*{\fill} \\}
    
    \begin{Verbatim}[commandchars=\\\{\}]
{\color{incolor}In [{\color{incolor}50}]:} \PY{n}{sns}\PY{o}{.}\PY{n}{jointplot}\PY{p}{(}\PY{n}{y}\PY{o}{=}\PY{l+s+s1}{\PYZsq{}}\PY{l+s+s1}{Age}\PY{l+s+s1}{\PYZsq{}}\PY{p}{,} \PY{n}{x}\PY{o}{=}\PY{l+s+s1}{\PYZsq{}}\PY{l+s+s1}{Pos\PYZus{}Ax\PYZus{}nodes}\PY{l+s+s1}{\PYZsq{}}\PY{p}{,} \PY{n}{data}\PY{o}{=}\PY{n}{short\PYZus{}lived}\PY{p}{,} \PY{n}{kind}\PY{o}{=}\PY{l+s+s1}{\PYZsq{}}\PY{l+s+s1}{kde}\PY{l+s+s1}{\PYZsq{}}\PY{p}{)}
\end{Verbatim}


\begin{Verbatim}[commandchars=\\\{\}]
{\color{outcolor}Out[{\color{outcolor}50}]:} <seaborn.axisgrid.JointGrid at 0x7f81cf075390>
\end{Verbatim}
            
    \begin{center}
    \adjustimage{max size={0.9\linewidth}{0.9\paperheight}}{output_41_1.png}
    \end{center}
    { \hspace*{\fill} \\}
    
    \paragraph{The most dense region falls around 50 for Age in both the
categories so is the same for both classes but as is evident from the
contour plot is that the number positive axill nodes is widely spread
for the short lived people which indicates an increased number of
positive lymph axill nodes for all the age
groups.}\label{the-most-dense-region-falls-around-50-for-age-in-both-the-categories-so-is-the-same-for-both-classes-but-as-is-evident-from-the-contour-plot-is-that-the-number-positive-axill-nodes-is-widely-spread-for-the-short-lived-people-which-indicates-an-increased-number-of-positive-lymph-axill-nodes-for-all-the-age-groups.}

    \subsection{Final Conclusion}\label{final-conclusion}

\paragraph{The given dataset is an imbalanced
dataset.}\label{the-given-dataset-is-an-imbalanced-dataset.}

\paragraph{It is not linearly seperable using a simple if-else
construct.}\label{it-is-not-linearly-seperable-using-a-simple-if-else-construct.}

\paragraph{The Number of Positive Axill nodes says a lot more about the
story as it shows the extent of the spread of the disease so intuitively
it is something which helps determining the life expectancy as it is
shown above in the data exploratory
analysis.}\label{the-number-of-positive-axill-nodes-says-a-lot-more-about-the-story-as-it-shows-the-extent-of-the-spread-of-the-disease-so-intuitively-it-is-something-which-helps-determining-the-life-expectancy-as-it-is-shown-above-in-the-data-exploratory-analysis.}


    % Add a bibliography block to the postdoc
    
    
    
    \end{document}
